\chapter{Internship Objectives}
\section{Learning Objectives}

Internships are educational and career development opportunities, providing practical experience in a field or discipline. They are structured, short-term, supervised placements often focused around particular tasks or projects with defined timescales. An internship may be compensated, non-compensated or some time may be paid. The internship has to be meaningful and mutually beneficial to the intern and the organization. It is important that the objectives and the activities of the internship program are clearly defined and understood. \\

Following are the intended objectives of internship training-

\begin{itemize}
\item Expose technical students to the industrial environment, which cannot be simulated in the classroom, thereby creating competent professionals for the industry.
\item Provide possible opportunities to learn understand and sharpen the real time technical /managerial skills required at the job.
\item Expose students to the current technological developments relevant to the subject area of training.
\item Utilize the experience gained from the ‘Industrial Internship’ in classroom discussions.
\item Create conducive conditions in the quest for knowledge and its applicability on the job.
\item Learn to apply their technical knowledge in real industrial situations.
\item Gain experience in writing technical reports/projects.
\item Expose students to the responsibilities and ethics of the engineering profession.
\item Familiarize students with the various materials, processes, products and their applications along with relevant aspects of quality control.
\item Promote academic, professional and/or personal development.
\item Expose the students to future employers.
\item Understand the social, economic and administrative considerations that influence the working environment of industrial organizations
\item Understand the psychology of the workers and their habits,attitudes and approach to problem solving.
\end{itemize}

